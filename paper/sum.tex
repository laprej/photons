%======================================================================
%----------------------------------------------------------------------
%               XX                              X
%                                               X
%               XX    XXX   XXX   XXX      XXX  X  XXXX
%                X   X   X X   X X   X    X   X X X
%                X   XXXXX XXXXX XXXXX    X     X  XXX
%                X   X     X     X     XX X   X X     X
%               XXX   XXX   XXX   XXX  XX  XXX  X XXXX
%----------------------------------------------------------------------
%  	         A SKELETON FILE FOR IEEE PAPER GENERATION
%----------------------------------------------------------------------
%======================================================================

% first, uncomment the desired options:
\documentclass[%
        %draft,
        %submission,
        %compressed,
        final,
        %
        %technote,
        %internal,
        %submitted,
        %inpress,
        %reprint,
        %
        %titlepage,
        notitlepage,
        %anonymous,
        narroweqnarray,
        inline,
        twoside,
        ]{ieee}
%
% some standard modes are:
%
% \documentclass[draft,narroweqnarray,inline]{ieee}
% \documentclass[submission,anonymous,narroweqnarray,inline]{ieee}
% \documentclass[final,narroweqnarray,inline]{ieee}

% Use the `endfloat' package to move figures and tables to the end
% of the paper. Useful for `submission' mode.
%\usepackage {endfloat}

% Use the `times' package to use Helvetica and Times-Roman fonts
% instead of the standard Computer Modern fonts. Useful for the 
% IEEE Computer Society transactions.
% (Note: If you have the commercial package `mathtime,' it is much
% better, but the `times' package works too).
%\usepackage {times}

% In order to use the figure-defining commands in ieeefig.sty...
\usepackage{ieeefig}

\usepackage{url}

\begin{document}

%----------------------------------------------------------------------
% Title Information, Abstract and Keywords
%----------------------------------------------------------------------
\title[Short Title]{%
       Accurate Radio Transmission Using Photon Mapping}

% format author this way for journal articles.
\author[SHORT NAMES]{%
      Justin M. LaPre
      \authorinfo{%
        J. LaPre is a Ph.D. student in the Department of Computer
        Science, Rensselaer Polytechnic Institute, Troy, NY, 12180, USA.}
      \and
      Mark E. Anderson
      \authorinfo{%
        M. Anderson is a Ph.D. student in the Department of Computer
        Science, Rensselaer Polytechnic Institute, Troy, NY, 12180, USA.}
  }

% format author this way for conference proceedings
%\author[SHORT NAMES]{%
%      Mark E. Anderson\member{Fellow}
%      \authorinfo{%
%      Department of Electrical Engineering\\
%      Some University, Somewhere CA, 90210, USA\\
%      Phone: (xxx) xxx-xxxx, email: xxx@xxxx.xxx.xxx}
%    \and
%      Justin M. LaPre\member{Senior Member}
%      \authorinfo{%
%      Department of Electrical Engineering...}
%  }

% specifiy the journal name
\journal{Advanced Computer Graphics, Spring 2011}

% Or, when the paper is a preprint, try this...
%\journal{IEEE Transactions on Something, 1997, TN\#9999.}

% Or, specify the conference place and date.
%\confplacedate{Troy, NY, USA, May 11, 2010}

% make the title
\maketitle               

% do the abstract
\begin{abstract}
%\section{Abstract}
NEW ABSTRACT
Accurate simulation of wireless radio signals is a challenging problem.  While primary effects can easily be calculated, second order effects are numerous and will often substantially alter the range of the radios.  Many modern simulators such as ns-3 use a simple, overly conservative calculation known as the Friis[Citation needed.] equation.
This work demonstrates a more accurate estimate of radio waves via photon mapping[Cite.].  By utilizing photon mapping we can predict where photons will travel as well as the objects they will strike.  By collecting the number of photons to strike an object (e.g. an antenna) the effective signal strength can be determined.  If the signal strength is above a pre-specified cutoff then the receiving node can hear the transmitter.
We show that this approach generates a higher fidelity estimate in most scenarios.  As expected, buildings and obstacles impact radio transmissions in ways that cannot be modeled by the Friis equation.  While the resolution of radio transmissions is increased, the amount of computation required for this result has grown considerably, I.e. A straightforward formula has been replaced with computing the paths of thousands of photons.

OLD ABSTRACT

Accurate simulation of wireless radio signals is a challenging problem.
While primary effects can be easily calculated, second order effects are
numerous and will often substantially alter the range of the radios.


Simulating wireless networks poses computational problems beyond their wired
counterparts.  Determining which nodes are within transmission range of a given
node's radio when sending is non-trivial.  Implementing this decision problem
in a naive manner will produce an O($n^2$) algorithm.  In a mobile ad-hoc
network this would need to be performed before every transmission.

Visible light and radio waves are essentially the same phenomenon;
we hope to use the abilities of modern GPUs to perform many computations
simultaneously to speed up this problem.  McGuire and Luebke
\cite{mcguire09imagespace} demonstrated real-time photon mapping
of complicated scenes utilizing GPU acceleration.  Schmitz et al.
\cite{Schmitz:2006:ERW:1164717.1164730}
\cite{Schmitz:2006:WPU:1163610.1163638}
improved the accuracy of their ns-2 model via photon mapping.  
Simulators such as ROSS \cite{ross} utilize the Transmission Line Matrix
\cite{Nutaro:2006:DEM:1138464.1138468} to determine which nodes are able
to receive transmissions.  We hope to improve upon both the accuracy and 
run-time of TLM in this work.

\end{abstract}
\vspace{5mm}
% do the keywords
\begin{keywords}
wireless networking, 802.11, simulation, photon mapping, ray tracing
\end{keywords}

\section{Schedule}
\subsection*{Week 1}
Background reading and research.  Learn CUDA / GLSL / OpenCL.

\subsection*{Week 2}
Most of the implementation should be done in this week.

\subsection*{Week 3}
More implementation, debugging, and code clean-up.

\subsection*{Week 4}
Write the paper.
% start the main text ...
%----------------------------------------------------------------------
% SECTION I: Introduction
%----------------------------------------------------------------------
\section{Introduction}

Wireless radio communication has become ubiquitous and, in fact, necessary for many purposes ranging from widespread wifi hotspots for web browsing to critical military applications.  Whether or not one node is capable of hearing another is clearly an essential question, the answer to which will determine the effective ability to communicate.
An oracle for deciding whether or not node A can hear node B would be useful in this situation.  Unfortunately, such an oracle does not exist and we must fall back on approximating an answer.  One such method adopted by ns-2 and ns-3 is the Friis[Cite.] equation.  The Friis equation assumes idealized conditions and that no other objects interfere with communications.  Clearly this is an over-simplification; many materials impact radio wave transmission in some way.
In this work we propose an improved method of approximating radio transmissions.  By modeling the radio communications as photons, determining the signal strength is simply a matter of gathering photons which impact a region of interest.  Photon mapping algorithm was developed by Jensen[Cite.] to improve upon deficiencies in classic ray tracing approaches.  
% do the bibliography:
\bibliographystyle{IEEEbib}
\bibliography{my-bibliography-file}

% where ``my-bibliography-file.bib'' is the name of the file with all the 
% BibTeX entries.

% do the biographies...
%\begin{biography}{Gregory L. Plett}
%  A bio with no face...
%\end{biography}

% If you want a picture with your biography, then specify the name of
% the postscript file in square brackets. That is, uncomment the
% following three lines and change the name of "face.ps" to the name of 
% your file.
%\begin{biography}[face.ps]{Gregory L. Plett}
%  A bio with a face...
%\end{biography}

%----------------------------------------------------------------------
% FIGURES
%----------------------------------------------------------------------
% There are many ways to include figures in the text. We will assume
% that the figure is some sort of EPS file.
%
% The outdated packages epsfig and psfig allow you to insert figures
% like: \psfig{filename.eps} These should really be done now using the
% \includegraphics{filename.eps} command.  
%
% i.e.,
%
% \includegraphics{file.eps}
%
% whenever you want to include the EPS file 'file.eps'. There are many
% options for the includegraphics command, and are outlined in the
% on-line documentation for the "graphics bundle". Using the options,
% you can specify the height, total height (height+depth), width, scale,
% angle, origin, bounding box "bb",view port, and can trim from around
% the sides of the figure. You can also force LaTeX to clip the EPS file
% to the bounding box in the file. I find that I often use the scale,
% trim and clip commands.
% 
% \includegraphics[scale=0.6,trim=0 0 0 0,clip=]{file.eps}
% 
% which magnifies the graphics by 0.6 (If I create a graphics for an
% overhead projector transparency, I find that a magnification of 0.6
% makes it look much better in a paper), trims 0 points off
% of the left, bottom, right and top, and clips the graphics. If the
% trim numbers are negative, space is added around the figure. This can
% be useful to help center the graphics, if the EPS file bounding box is
% not quite right.
% 
% To center the graphics,
% 
% \begin{center}
% \includegraphics...
% \end{center}
% 
% I have not yet written good documentation for this, but another 
% package which helps in figure management is the package ieeefig.sty,
% available at: http://www-isl.stanford.edu/people/glp/ieee.shtml
% Specify:
% 
%\usepackage{ieeefig} 
% 
% in the preamble, and whenever you want a figure,
% 
%\figdef{filename}
% 
% where, filename.tex is a LaTeX file which defines what the figure is.
% It may be as simple as
% 
% \inserteps{filename.eps}
%
% or
% \inserteps[includegraphics options]{filename.eps}
% 
% or may be a very complicated LaTeX file. 

\end{document}
